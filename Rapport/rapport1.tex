\documentclass[a4paper, titlepage, oneside, 12pt]{article}%      autres choix : book  report

\usepackage[utf8]{inputenc}%           gestion des accents (source)

\usepackage[T1]{fontenc}%              gestion des accents (PDF)

\usepackage[francais]{babel}%          gestion du français

\usepackage{textcomp}%                 caractères additionnels

\usepackage{mathtools,  amssymb, amsthm}% packages de l'AMS + mathtools

\usepackage{lmodern}%                  police de caractère

\usepackage{geometry}%                 gestion des marges

\usepackage{graphicx}%                 gestion des images

\usepackage{xcolor}%                   gestion des couleurs

\usepackage{array}%                    gestion améliorée des tableaux

\usepackage{calc}%                     syntaxe naturelle pour les calculs

\usepackage{titlesec}%                 pour les sections

\usepackage{titletoc}%                 pour la table des matières

\usepackage{fancyhdr}%                 pour les en-têtes

\usepackage{titling}%                  pour le titre

\usepackage[framemethod=TikZ]{mdframed}% print frames

\usepackage{caption}%                  for captionof

\usepackage{listings}

\usepackage{enumitem}%                 pour les listes numérotées

\usepackage{microtype}%                améliorations typographiques

\usepackage{csvsimple}%                convertir un fichier .csv en tableau

\usepackage{adjustbox}%                centrer les tableaux

\usepackage{hyperref}%                 gestion des hyperliens

\usepackage[french]{algorithm2e}
\hypersetup{pdfstartview=XYZ}%         zoom par défaut

%%%%%%%%%%%%%%%%%%%%%% CONFIGURATION %%%%%%%%%%%%%%%%%%%%%%%%%%

\mdfdefinestyle{MyFrame}{%
    linecolor=black,
    outerlinewidth=0pt,
    roundcorner=10pt,
    innertopmargin=\baselineskip,
    innerbottommargin=\baselineskip,
    innerrightmargin=20pt,
    innerleftmargin=20pt,
    backgroundcolor=black!10!white}

\captionsetup[lstlisting]{labelformat=empty}

%%%%%%%%%%%%%%%%%%%%%%%%%%%%%%%%%%%%%%%%%%%%%%%%%%%%%%%%%%%%%%%

\newtheorem{prop}{Proposition}

\hypersetup{                    % parametrage des hyperliens
    colorlinks=true,                % colorise les liens
    breaklinks=true,                % permet les retours à la ligne pour les liens trop longs
    urlcolor= red,                 % couleur des hyperliens
    linkcolor= blue,                % couleur des liens internes aux documents (index, figures, tableaux, equations,...)
    citecolor= green                % couleur des liens vers les references bibliographiques
    }
%%%%%%%%%%%%%%%%%%%%%% COMMANDES PL %%%%%%%%%%%%%%%%%%%%%%%%%%
\newcommand\boldmin{\mathop{\mathbf{min}}}
\newcommand\boldmax{\mathop{\mathbf{max}}}



% I don't know how it works but it does !
\def\foo#1#2{%
\newenvironment{variables}
{\paragraph{Variables :}
#1{2}}
{#2}}
\expandafter\foo
  \csname alignat*\expandafter\endcsname
  \csname endalignat*\endcsname


% \newenvironment{variables}
% {  \paragraph{Variables :}
  
%   \alignat{2} }
% {  \endalignat }

\newenvironment{fonctionobj}
{ \paragraph{Fonction objectif :}


  }
{  }

\def\cons#1#2{%
\newenvironment{contraintes}
{\paragraph{Contraintes :}
#1{2}}
{#2}}
\expandafter\cons
  \csname alignat*\expandafter\endcsname
  \csname endalignat*\endcsname


\newcommand{\variable}[4]{\underbrace{#1}_{\mathclap{\text{#4}}} \in #2 &\ #3 \\}
\newcommand{\fobj}[2]{\begin{alignat*}{2} #1 &\qquad \text{\footnotesize (#2)} \end{alignat*}}
\newcommand{\constraint}[3]{#1 &,\ & #2 & 
\if\relax\detokenize{#3}\relax
\\
\else
\qquad \text{\footnotesize \textcolor{blue}{\textit{#3}}} \\ 
\fi}
%%%%%%%%%%%%%%%%%%%%%%%%%%%%%%%%%%%%%%%%%%%%%%%%%%%%%%%%%%%%%%%




\title{Rapport pour le projet de MOGPL}
\date{\today}
\author{TRAORE BASSIRO OUREMI}

 % Modèle pour écrire un programme linéaire
 % \begin{alignat*}{2}
 %    \text{minimize }   & \sum_{i=1}^m c_i x_i + \sum_{j=1}^n d_j t_j\  \\
 %    \text{subject to } & \sum_{i=1}^m a_{ij} + e_j t_j \geq g_j &,\ & 1\leq j\leq n\\
 %                       & f_i x_i + \sum_{j=1}^n b_{ij}t_j \geq h_i\ &,\ & 1\leq i\leq m\\
 %                       & x\geq 0,\ t_j\geq 0\ &,\ & 1\leq j\leq n,\ 1\leq i\leq m
 %  \end{alignat*}


\begin{document}

\maketitle
{
  \hypersetup{linkcolor=black}
  \tableofcontents
}

\newpage

\section{modélisation PNLE}
\subsection{Question 10}
La définition du programme linéaire $\mathcal{P}_0$ est donnée dans l'encadré ci-dessous :
\begin{mdframed}[style=MyFrame]

%$P^{w_i}$
\begin{contraintes}
    \constraint{\sum_{k=j}^{j+st-1} z^t_{i,k} \leq st}{ 0\leq j\leq M-1,0\leq i\leq N-1,1\leq t\leq nbsequence de la colonne j}{$1$ pour les colonnes}
     \constraint{\sum_{k=i}^{i+st-1} y^t_{k,j} \leq st}{ 0\leq j\leq M-1,0\leq i\leq N-1,1\leq t\leq nbsequence de la ligne i}{$1$ pour les lignes}
    %\constraint{\sum_{j=1}^n x_{i,j} =1}{ 1\leq i\leq n}{$1$ bien par agent}
    %\constraint{z_i(x) +d_{k_i} -(d_{k_{seuil}^+}-d_{k_{seuil}^-})\geq 0}{ 1\leq k,i\leq n}{}
\end{contraintes}
\end{mdframed}
\captionof{lstlisting}{Programme linéaire $\mathcal{L}_k$}

\subsection{Question 11a}
Formulez une contrainte qui exprime le décalage nécessaire entre le début des blocs s t et s t+1
de la ligne li:
\begin{mdframed}[style=MyFrame]

%$P^{w_i}$
\begin{contraintes}
    \constraint{z^t_{i,j}+z^{t+1}_{i+st,j} \leq 1}{ 0\leq j\leq M-1,0\leq i\leq N-1,1\leq t\leq nbsequence de la colonne j}{$1$ pour les colonnes}
    \constraint{y^t_{i,j}+y^{t+1}_{i,j+st} \leq 1}{ 0\leq j\leq M-1,0\leq i\leq N-1,1\leq t\leq nbsequence de la ligne i}{$1$ pour les lignes}
    % \constraint{\sum_{k=i}^{i+st-1} y^t_{k,j} \leq st}{ 0\leq j\leq M-1,0\leq i\leq N-1,1\leq t\leq nbsequence de la ligne i}{$1$ pour les lignes}
    %\constraint{\sum_{j=1}^n x_{i,j} =1}{ 1\leq i\leq n}{$1$ bien par agent}
    %\constraint{z_i(x) +d_{k_i} -(d_{k_{seuil}^+}-d_{k_{seuil}^-})\geq 0}{ 1\leq k,i\leq n}{}
\end{contraintes}

\end{mdframed}

\subsection{Question 11b}
contrainte suplementaires
\begin{mdframed}[style=MyFrame]

%$P^{w_i}$
\begin{contraintes}
    \constraint{\sum_{i=0}^{M-1} \sum_{t=1} z^t_{i,j} \leq \sum_{t=1} st}{ 0\leq j\leq N-1}{$1$ pour les colonnes}
    \constraint{\sum_{j=0}^{N-1} \sum_{t=1} y^t_{i,j} \leq \sum_{t=1} st}{ 0\leq i\leq M-1}{$1$ pour les lignes}
    %\constraint{\sum_{k=j}^{j+st-1} z^t_{i,k} \leq st}{$1$ pour les colonnes}
    %\constraint{y^t_{i,j}+y^{t+1}_{i,j+st} \leq 1}{ 0\leq j\leq M-1,0\leq i\leq N-1,1\leq t\leq nbsequence de la ligne i}{$1$ pour les lignes}
    % \constraint{\sum_{k=i}^{i+st-1} y^t_{k,j} \leq st}{ 0\leq j\leq M-1,0\leq i\leq N-1,1\leq t\leq nbsequence de la ligne i}{$1$ pour les lignes}
    %\constraint{\sum_{j=1}^n x_{i,j} =1}{ 1\leq i\leq n}{$1$ bien par agent}
    %\constraint{z_i(x) +d_{k_i} -(d_{k_{seuil}^+}-d_{k_{seuil}^-})\geq 0}{ 1\leq k,i\leq n}{}
\end{contraintes}


\end{mdframed}
%\captionof{lstlisting}{Programme linéaire $\mathcal{L}_k$}


\subsection{Question 15}
Résolvez les instances 1.txt à 16.txt avec un timeout de 2 minutes. Donnez les temps de
calcul dans un tableau.
\begin{table}
\centerline{
\csvautotabular {data/question15/question15a.csv}
}
\caption{Pour la résolution du PLNE des instances 1 à 16 }
\label{question15}
\end{table}

\end{document}

%%% Local Variables:
%%% mode: latex
%%% TeX-master: t
%%% End:
